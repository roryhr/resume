\documentclass[10pt,letterpaper]{article}
\usepackage[letterpaper, margin=1.0in,  bottom=.8in]{geometry}
\usepackage[pdfborder={0 0 0}]{hyperref}
\usepackage{mdwlist}
\pagestyle{empty}
\setlength{\tabcolsep}{0em}

\newenvironment{indentsection}[1]%
{\begin{list}{}%
	{\setlength{\leftmargin}{#1}}%
	\item[]%
}
{\end{list}}

% opposite of above; bump a section back toward the left margin
\newenvironment{unindentsection}[1]%
{\begin{list}{}%
	{\setlength{\leftmargin}{-0.5#1}}%
	\item[]%
}
{\end{list}}

% format two pieces of text, one left aligned and one right aligned
\newcommand{\headerrow}[2]
{\begin{tabular*}{\linewidth}{l@{\extracolsep{\fill}}r}
	#1 &
	#2 \\
\end{tabular*}}


\begin{document}

{\raggedright \LARGE \bf Rory Hartong-Redden\\}

{\raggedleft 
Menlo Park, CA \textbar\/ 
roryhr@gmail.com \textbar\/ 
Cell: 925.297.9484 \textbar\/ 
\href{roryhr.github.io}{roryhr.github.io} \textbar\/   
\href{https://github.com/roryhr}{github.com/roryhr}\\}
\hrule

\subsection*{Summary}
\begin{centering}  
research scientist, machine learning aka deep learning aka AI, python developer, data scientist, physicist\\
\end{centering}

\hrule
\subsection*{Skills Summary}
\begin{indentsection}{\parindent}
\begin{tabular}{p{0.5\linewidth}   p{0.5\linewidth} } 
	\textbf{Languages:}  Python, R	
	& \textbf{ML:} Neural Networks, Gradient Boosted Machines \\

	\textbf{Data:} SQL, Postgres, HDF5
		& \textbf{Dev:}  Git, Heroku, PyCharm, AWS \\  
	
	\multicolumn{2}{l}{\textbf{Python Tools:} Conda, Keras, matplotlib, NumPy, OpenCV, Pandas, scikit-learn, SQLAlchemy} \\
\end{tabular}
\end{indentsection}

\hrule
\subsection*{Work Experience}
\begin{itemize}
	\parskip=-0.1em
	\item
	\headerrow
	{\textbf{Allstate}}
	{Menlo Park, CA}
	\headerrow
	{\emph{Lead Research Analyst}}
	{\emph{Jul 2015--Present}}
	\begin{itemize*}
		\item Working on a secret project (to be published) in the autonomous vehicle and safety space
	\end{itemize*}
\end{itemize}

\begin{itemize}
	\parskip=-0.1em
	\item
	\headerrow
		{\textbf{Startup.ML}}
		{San Francisco, CA}
	\headerrow
		{\emph{Machine Learning Fellow}}
		{\emph{Dec 2015--Apr 2016}}
	\begin{itemize*}
		\item Developed a FinTech data pipeline for algorithmic currency trading using machine learning
		\item Lead the integration of an open-source Python backtesting and trading platform
		\item Research in Reinforcement Learning for autonomous trading
	\end{itemize*}
\end{itemize}

\begin{itemize}
	\parskip=-0.1em
	\item
	\headerrow
		{\textbf{Harold Washington College}}
		{Chicago, IL}
	\headerrow
		{\emph{Adjunct Faculty}}
		{\emph{Feb 2015--May 2015}}
	\begin{itemize*}
		\item Presented complex concepts in Astronomy at the level of the general public
	\end{itemize*}
\end{itemize}


\hrule
\subsection*{Projects}
\begin{itemize}
	\parskip=-0.1em
%	\item
%	\headerrow
%		{\textbf{Kaggle}}{\emph{May 2015--Present}}
%	\begin{itemize*}
%		\item Competing in the ``Distracted Driver" Kaggle competition with a deep convolutional neural network written in Keras/TensorFlow 
%	\end{itemize*}	
	\item
	\headerrow
		{\textbf{Master's Thesis}}
		{UC Santa Barbara}	\\
	\headerrow
		{\emph{Krechetnikov Fluid Physics Lab}}
		{\emph{Dec 2013--Jun 2014}}
	\begin{itemize*}
		\item Incorporated a recent image processing technique for cheap 3D high speed mm-resolution measurement over a surface area of $225 \,\mathrm{cm}^2$ 		\href{https://github.com/roryhr/profilometry}{[profilometry repository]}
	\end{itemize*}
	
	\item
	\headerrow
		{\textbf{Bachelor's Thesis: Drop Splash Experiment}}
		{Santa Barbara, CA} \\
	\headerrow
		{\emph{Krechetnikov Fluid Physics Lab, Dept.~of Mechanical Engineering}}
		{\emph{Jul 2009--Oct 2010}}
	\begin{itemize*}
%		\item Investigated the physics of splashes that occur when a liquid droplet impacts a wetted surface
% 		\item Performed stereo triangulation in MATLAB, reduced the  the 3D data, and searched for patterns using my theory of pattern identification 
% 		\href{https://github.com/roryhr/drop_splash}{[drop\_splash repository]}
		\item Published a peer-reviewed article\footnote{Hartong--Redden, R.\ \& Krechetnikov, R. {\em Pattern identification in systems S(1) symmetry.} Phys.\ Rev.\ E. 2011.} on the experimental and theoretical advances I developed that may have solved a 100-year puzzle in fluid dynamics
	\end{itemize*}
	\item
	\headerrow
		{\textbf{Transient Optical Sky Survey}}
		{UC Santa Barbara}	\\
	\headerrow
		{\emph{Lubin Lab, Dept.~of Physics}}
		{\emph{Sep 2008--Jun 2009}}
	\begin{itemize*}
		\item Image processing with a Matlab/C data pipeline that analyzed 1GB of images taken each night
	\end{itemize*}
\end{itemize}


\hrule
\subsection*{Education}
\begin{itemize}
	\parskip=-0.1em
	\item 
	\headerrow
		{\textbf{University of California, Santa Barbara}}
		{Santa Barbara, CA}
	\headerrow
		{\emph{MS  Mechanical Engineering}}
		{\emph{Dec 2014}}
%	\begin{itemize*}
%%		\vspace{-0.2em}
%		\item Thesis: {\em Experimental apparatus for the study of Faraday waves on time-varying domains}
%	\end{itemize*}
	\item 
	\headerrow
		{\textbf{Northwestern University}}
		{Evanston, IL}
	\headerrow
		{\emph{PhD Physics Candidate}}
		{\emph{Sept 2010--Mar 2012}}
	\item 
	\headerrow
		{\textbf{University of California, Santa Barbara}}
		{Santa Barbara, CA}	\\
	\headerrow
		{\emph{BS Physics \&  BS Mechanical Engineering}}
		{\emph{June 2010}}
	\begin{itemize*}
%		\item  Thesis: {\em Experimental and theoretical study of pattern identification in physical systems with circular symmetry}
		\item Graduated with honor in both degrees, Dean's list 11/12 quarters, GPA: 3.7/4.0
		\item Member: Tau Beta Pi engineering honor society
	\end{itemize*}
\end{itemize}

\end{document}
