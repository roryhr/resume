\documentclass[10pt,letterpaper]{article}
\usepackage[letterpaper, margin=1.0in,  bottom=.8in]{geometry}
\usepackage{mdwlist}
\pagestyle{empty}
\setlength{\tabcolsep}{0em}

\newenvironment{indentsection}[1]
{\begin{list}{}%
	{\setlength{\leftmargin}{#1}}
	\item[]%
}
{\end{list}}

% format two pieces of text, one left aligned and one right aligned
\newcommand{\headerrow}[2]{
\begin{tabular*}{\linewidth}{l@{\extracolsep{\fill}}r}
		#1 &
		#2 \\
	\end{tabular*}
}

\newcommand{\jobitem}[4]{\item \headerrow{\textbf{#1}}{#2}
\headerrow{\emph{#3}}{\emph{#4}}}

\begin{document}

{\raggedright \LARGE \bf Rory Hartong-Redden}

{\raggedleft 
\today \/ \textbar
\/ Boulder, CO \textbar
\/ roryhr@gmail.com \textbar
\/ linkedin.com/in/rory-hartong-redden \textbar
\/ roryhr.com
\\
}


\subsection*{Summary}
\begin{centering}
Experienced data scientist with accomplishments across scientific research, data engineering, and software engineering. 
Core competencies: Python, ML, and statistical analysis, complemented by 3 years of management experience. 
\end{centering}

\subsection*{Skills}

\begin{indentsection}{\parindent}
\begin{tabular}{p{0.5\linewidth}   p{0.5\linewidth}} 
	Programming Languages: Python, SQL, R, Shell
	& DevOps: GitHub, AWS, Docker,  CI/CD, CircleCI \\

	\multicolumn{2}{l}{Data Processing: Postgres, Spark, Hadoop, S3, Airflow, Databricks} \\
	\multicolumn{2}{l}{
		Python Tools: pandas, scikit-learn, Flask, XGBoost, TensorFlow, PySpark, Matplotlib, Jupyter
		}
\end{tabular}
\end{indentsection}

\hrule
\subsection*{Professional Experience}
\begin{itemize}
	\jobitem{Allstate}{Menlo Park, CA}
		     {Lead Data Scientist}{Jan 2025--Present}
	\begin{itemize*}
		\item Consulting within the data science organization to move workloads onto AWS and incorporate best practices like CI/CD, testing, and using infrastructure as code
	\end{itemize*}
\end{itemize}

\begin{itemize}
	\jobitem{Fast Radius (SyBridge)}{Boulder, CO}
		     {Senior Data Scientist / Technical Manager}{Aug 2021--May 2024}
	\begin{itemize*}
		\item Led a 3-person data science team, partnering with product and engineering teams, and shipped key software products like part analysis that were featured in the FSRD IPO presentation
		\item Ensured API uptime of our revenue-critical service, facilitating \$10M in monthly instant quotes
		\item Trained and deployed a random forest regression model of CNC cycle time for CNC instant quoting, providing the majority of the revenue of our site
		\item Predicted shipping costs using mixed integer programming to pack boxes as input to the UPS API
%        \item Performed time-series analysis on manufacturing IoT data to support R\&D efforts
%    \item Used SQL to extract and analyze data, monitor model performance, create dashboards, and answer business questions
    \end{itemize*}
\end{itemize}

\begin{itemize}
	\jobitem{Fast Radius}{Chicago, IL}
		     {Data Scientist}{Feb 2020--Aug 2021}
	\begin{itemize*}
		\item Tech stack: Python, Flask, Docker, AWS ECS, Datadog, Metabase
	    \item Founding data scientist: developed, deployed, and maintained the first cost-prediction regression models, contributing to the early growth of the platform
	    \item Contributed to the ``Manufacturing and Development Platform" patent, laying the groundwork for the company's software architecture
		\item In the startup culture I dipped into the Elixir backend, JavaScript frontend, or Terraform infrastructure to fix bugs and remove blockers
	\end{itemize*}
\end{itemize}
\begin{itemize}
	\jobitem{runtastic}{Linz, Austria}
		     {Data Engineer}{Oct 2018--Sep 2019}
	\begin{itemize*}
		\item Tech stack: Python, SQL, Spark, Hadoop, Flume, Oozie, Hive
		\item Designed and developed a ``People You Might Know" feature using Python, Spark, and Hadoop, improving user retention
		\item Built an ETL pipeline to anonymize customer data for GDPR compliance
		\item Led the design and development of a data exchange prototype using Apache Kafka and AWS S3
	\end{itemize*}
\end{itemize}

\begin{itemize}
	\jobitem{Allstate}{Menlo Park, CA}
		     {Research Analyst}{Jul 2016--Sep 2018}
	\begin{itemize*}
		\item Tech stack: Python, pandas, TensorFlow, Spark, Julia, PostGIS
	    \item Developed and trained machine learning models using Python, pandas, TensorFlow, and Spark to predict risk using customer GPS data, public crash data, and synthetic data
		\item Co-authored a research paper with the Stanford Intelligent Systems Lab: ``Real-time Prediction of Intermediate-Horizon Automotive Collision" 
%	        \item Analyzed large datasets and communicated findings to stakeholders
	\end{itemize*}
\end{itemize}


\hrule
\subsection*{Education}
\begin{itemize}
	\jobitem{University of California, Santa Barbara}{Santa Barbara, CA}
		    {MS  Mechanical Engineering}{Dec 2014}
	\begin{itemize*}
		\item Tech stack: MATLAB, SolidWorks, \LaTeX
		\item Thesis research: Designed and built an experiment to measure Faraday waves using a novel image processing technique for 3D high-speed mm-resolution measurement over a surface area of $225 \,\mathrm{cm}^2$ 
	\end{itemize*}
	\jobitem{University of California, Santa Barbara}{Santa Barbara, CA}	
		    {BS Physics \&  BS Mechanical Engineering}{June 2010}
\end{itemize}
\end{document}